\section{Block Diagram Version 2}
		A system block diagram was redesigned from scratch after considering the components that we would use for our implementation. After consulting professor Manuel Jimenez and director Raul Zapata, we determined components that will be used in the project and ended up with the following system block diagram:

		
		\begin{figure}[H]
			\centering
				\includegraphics[scale=0.30]{img/blockdiagramv2}
			\caption{The second version of the AeroBal block diagram.}
		\end{figure}

		\subsection{Material/Component search.}

			The following components were determined to be feasible for use in the project. The links to the webpages from which we will obtain or build these components are all listed at \url{aerobalmicro2.blogspot.com} as a post on September 11.

			\begin{itemize}
				\item Anemometer
				\begin{itemize}
					\item To determine the speed of the wind inside the tunnel.
					\item A computer fan may be used as an Anenometer. Due to initial resistance, the fan may have to be \'jump-started\' to eliminate initial resistance to move.
					\item Interfacing can be achieved by using digital general purpose IO ports from the MCU.
				\end{itemize}
				\item Barometic Pressure Sensor
				\begin{itemize}
					\item Module Model BMP085 will be used.
					\item Interfacing can be achieved using I2C protocol.
				\end{itemize}
				\item Wind Direction Sensor
				\begin{itemize}
					\item For the wind direction sensor (or wind vane), a potentiometer along with the tailvane may help in determining the direction as suggested by Manuel.
					\item An ADC may be used for interfacing with this component.
				\end{itemize}
				\item Strain Gauges
				\begin{itemize}
					\item Found to be the ones used by weight scales for mail packages.
					\item An ADC will be used for interfacing with this component.
				\end{itemize}
				\item LCD
				\begin{itemize}
					\item As an LCD we will use one based on the model JHD204A that uses a SPLC780D Controller.
					\item UART protocol may be used with the Serial LCD Backpack Module that will convert the signal to de Parallel Bus that the LCD uses.
				\end{itemize}
				\item Bluetooth Module
				\begin{itemize}
					\item The model BLE112 from Bluegiga will be used.
					\item Interfacing with this component can be achieved by using the UART protocol.
				\end{itemize}
				\item Humidity and Temperature Sensor
				\begin{itemize}
					\item A humidity sensor and a temperature sensor come as a single module in the package DHT11.
					\item Interfacing can be achieved with general purpose IO ports.
				\end{itemize}
				\item Buzzer, LEDs and Buttons.
				\begin{itemize}
					\item Provided for a robust user interface in the hardware.
					\item We can use common buzzers, as well as common LEDs and Buttons.
				\end{itemize}
				\item Relays
				\begin{itemize}
					\item The actuator that controls the windows for adjusting the wind speed (or the window actuator) is based on the model: MP-485. The control is based on a switch (Snap Acting or Floating Switch as in the specs).
					\begin{figure}[H]
						\centering
							\includegraphics[scale=0.60]{img/window-actuator}
						\caption{Actuator for controlling the window. This allows for control of the speed of the wind.}
					\end{figure}
					\item Relays can be used to achieve the logic of this tri-state switch in the following manner, and this approach will be used if approved:
					\begin{figure}[H]
						\centering
							\includegraphics[scale=0.60]{img/relay-circuit}
						\caption{A possible implementation of the circuit to enable control of the window actuator}
					\end{figure}

				\end{itemize}
			\end{itemize}
\newpage