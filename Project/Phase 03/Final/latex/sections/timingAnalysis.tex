\section{Timing Analysis}
	
	\subsection{Minimum Pulse Width}

		The minimum pulse width for a system is obtained by determining the component which works with the fastest frequency. This determines the constraints for the processor in terms of clock speed and frequency division. For the Aerobal system, the components that require interface with a specific/given frequency are:

		\begin{itemize}
			\item BMP-085 - Barometric Pressure Sensor
			\item US-1881 - Hall-Effect Sensor for Fan Anemometer
			\item EGBT-045 - Bluetooth Module
			\item C2004A - LCD HD44780 Based
			\item DHT11 - Humidity and Temperature Sensor
		\end{itemize}


		\subsubsection{BMP085 - Barometric Pressure Sensor}
			The Barometric Pressure Sensor uses as its fastest interface the I\textsuperscript{2}C. The maximum speed it attains is 3.4Mhz. The minimum pulse witdh would be ~0.29us.

			$$Maximum Frequency: I^2C @ 3.4Mhz$$

			$$Minimum Pulse Width = 1/(3.4MHz) = ~0.29us$$

	
		\subsubsection{US-1881 - Hall-Effect Sensor for Fan Anemometer}
			A fan will be used to work as an anemometer for the system. The fan only requires the blades to spin to be detected by a sensor that is contained within them. This sensor sends two pulses per rotation. Hence the rotation at which the anemometer turns determines the working frequency of the component.

			
				Rotation speed in fan datasheets = 4500 rpm @ 2 pulses per revolution
			
				= 9000 pulses per minute
			
				= 9000ppm/60 = 150pps max. = 150Hz Maximum Frequency

			$$Minimum Pulse Width = 1/(150Hz) = 6.67 ms$$


		\subsubsection{EGBT-045 - Bluetooth Module}
			To achieve UART Communication the clock of the MCU is set to 8, 16, 32, and 64 times higher than the baud rate.This module can be configure to read a baud rate of 921600 although 9600 is the default. 

			However, the Tiva MCU does not have documented Baud Rate error table but has a Performance Limitation table that details the attainable and successful baud-rate communication using 9600 bps to 115200 bps. The minimum frequency specified for which the maximum baud is successful is 20 Mhz. Therefore the only way we can use the highest baud is we configure the MCU to work at 20 MHz.

			Using the default, we can achieve communication although the worst-case configuration is demostrated here.

			$$ Acceptable Baud: 9600bps $$
			
				Tiva configuration for 9600bps requires 8Mhz
			$$ Minimum Pulse Width: (1/8MHz) = 125 ns $$

			$$ Worst-Case Baud: 115200bps $$
			
				Tiva configuration for 115200bps requires 20Mhz
			$$ Minimum Pulse Width: (1/20MHz) = 50 ns $$

			Due to the limitation of the Baud Rate error table, the error that the clock causes on the baud rate cannot be shown here.

		\subsubsection{2004A - LCD: HD44780 Based}
			The specifications inidicate that the MCU must correspond/conform to a 2Mhz bus interface. This means the frequency for interfacing is 2Mhz. It can be slower than this however.

				$$ Maximum frequency: 2 MHz $$
				$$ Minimum Pulse Width: 1/(2Mhz)= 0.5 us $$


		\subsubsection{DHT11 - Humidity and Temperature Sensor}
			The communication with the sensor requires a provided digital algorithm to read the data from it. The algorithm specifies signals that must be sent for a period of time.

			\begin{itemize}
				\item 18ms Start Signal ACK
				\item 20-40us MCU Pull Up Voltage ACK
				\begin{itemize}
					\item 80 us low-v-level response signal.
					\item 80 us high-v-level preparation signal.
					\begin{itemize}
						\item 50 us low-v-level means start bit transmit.
						\item 26-28 us high-v-level is 0.
						\item 70 us high-v-level is 1.
						\item 50 us	low-v-level means end.
					\end{itemize}
				\end{itemize}
			\end{itemize}

			Hence, the minimum pulse width needed would be:

			$$ Minimum Pulse Width = 20us $$
			$$ Maximum Frequency = 1/(20us) = 50 KHz $$

	\subsection{Real-Time Clocks}
		This system requires the use of a real-time clock to be able to keep track of time in human units since calculations are made that use units of time (seconds), for example, the speed of the wind. which is measured
		in either kilometers per second or miles per second.

		An external 32.768KHz clock can be used as input to allow the use of real-time clocks in the Tiva Microprocessor.

	\subsection{Baud-Error}
		Due to the limitation of the Baud Rate error table not being available for the Tiva (substituted by UART Performance Limitation), the error that the clock causes on the baud rate cannot be shown here.

	\subsection{Baud-Rate Selection}
		The default configuration of the Bluetooth module will be used since this one is sufficient for tranferring small amounts of data (bytes). The data to transfer is not big in size, but only measurements taken from the sensors of of the system.

	\subsection{System Clock Selection}
		Hence the \textbf{worst-case} minimum pulse width is: 

					$$ 50ns $$

		\noindent \textbf{Source}: Tiva w/ Bluetooth Module @ 115.2 Kbps Baud
		requires 20 Mhz Frequency.

		\noindent Processor must have a minimum frequency of:
			
			$$ 20 Mhz < 80Mhz  = Tiva Frequency $$


		\noindent A \textbf{functional} minimum pulse width:

		\noindent \textbf{Source:} Tiva w/ Bluetooth Module @ 9.6 Kbps Baud requires 8 Mhz Frequency.

		\noindent This forces the processor to have a minimum frequency of:
				
				\textbf{$$ 8 Mhz <  80Mhz = Tiva Frequency $$}

		\noindent Tiva gives us the benefit of having up to 80 Mhz to work with. Hence the clock can be
		lowered to about 25Mhz without affecting the communication  and minimum pulse width requirement and
		still satistying the worst-case requirements.

\newpage