\section{System Specifications}

		In this section we provide some of the specifications of all of the major components in \textbf{AeroBal}. They are included an interfaced in a strategic way to provide the user the best possible experience when conducting research on their various testing models.
		
		\subsection{MCU: Tiva C Series TM4C123G}
		
		Our selected MCU is the TM4C123G from Texas Instruments, widely known as the Tiva C Series launchpad. Our two biggest reasons for selecting it were the number of available pins (40) and processing power, both at a low price (\$12.99). Figure 8 shows an image of our MCU.
		
		% from https://estore.ti.com/Assets/ProductImages/2013-04-12_Tiva_Launchpad_flat.jpg
		\begin{figure}[H]
			\centering
				\includegraphics[scale=2.0]{img/tivaCSeriesPNG}
			\caption{The TM4C123G Tiva C Series Launchpad}
		\end{figure}
		
		Some of the notable features of our the Tiva are the following:
		
		\begin{itemize}
		 \item ARM Cortex-M4F processor core.
		 \item 256 KB flash memory and 2 KB of EEPROM.
 		 \item 8 UART Ports, 4 SSI, and 4 I2C Ports.
 		 \item 2 PWM modules.
 		 \item 2 12-bit ADC modules.
 		 \item 1 3.3 V Voltage Source.
 		 \item 1 5 V Voltage Source.
        \end{itemize}
        
        Given our large amount of interfaced sensors and devices this microcontroller provided us with the ability to interface multiple components at the same time. The ARM Cortex processor core has also given us great flexibility for our large number of computations with all the data that we process through the system.
		
		
		\subsection{Strain Gauges}
		
		The strain gauges are the most important part of the system besides the Tiva C. They provide the user the data of how much force is being exerted to the testing model. Naturally, these forces will vary depending on the aerodynamic characteristics of the object being tested. \\
		
		\begin{figure}[H]
			\centering
				\includegraphics[scale=0.7]{img/Capture}
			\caption{Strain Gauges inside the tunnel}
		\end{figure}
		
		For our system, we selected gauges that can measure up to 20 Kg of forces. These are surplus to the requirements and should provide a user testing small objects (such as toy cars) more than enough flexibility to test their devices. \\
		
		These strain gauges were placed strategically to measure 3 axis: X, Y, and Z. The X and Y axis require two strain gauges, while the Z axis only requires one. \\
		
		The gauges require 5 V each, and their data is read using the ADC in the Tiva. Initially, the signal provided is not high enough. Therefore, they go through an amplification process using INA 129 Operational Amplifiers from Texas Instruments. These amplifiers are extremely cheap, and make the strain gauges work fine after building a small and simple circuit amplifying their signals.
		
		\subsection{Bluetooth Connection}
		
		Bluetooth connection to external devices, such as cell phones and tablets, is a feature that users should find more than handy. Being able to see the data they want through a wireless connection using their tablets or cell phones is one of the big steps we wanted to bring the user more comfort.  \\
		
		For AeroBal, we used the EGBT Bluetooth Serial Port Module. It has UART Connectivity and programmable baud rate. The main idea of this bluetooth module was to allow wireless connectivity from the embedded system to a tablet or cell phone with the Android application. 
		
		\begin{figure}[H]
			\centering
				\includegraphics[scale=0.35]{img/bluetoothModule}
			\caption{The HC-05 (left) interfaced with the Tiva C (right)}
		\end{figure}
		
		Some of the main hardware features in the HC-05 are:
		
		\begin{itemize}
		 \item -80dBm sensitivity
		 \item Low Power 1.8V Operation
		 \item UART Interface with programmable baud rate
		 \item Integrated antenna
        \end{itemize}
        
       	The integrated antenna has also eased us of working with that problem as well. 	
		
		\subsection{Temperature and Humidity Sensor}
		
		The pressure and humidity sensors are included to give more useful data to the users about the conditions in the current environment. In our system, we used the DHT11. It connects to an 8-bit microcontroller and delivers high-precision measurements. All of this in a very cost-effective package. 
		
		\begin{figure}[H]
			\centering
				\includegraphics[scale=0.45]{img/DHT11}
			\caption{DHT11 sensor (blue) interfaced with Tiva C and LCD Screen}
		\end{figure}
		
		The DHT11 runs with a voltage of 5V using the Tiva C's integrated 5V source.
		
		
		\subsection{LCD Screen}
		
		A good LCD screen was a vital part of our system to provide output of experiments. Although an Android Application is a great feature for the user, an LCD screen provides a hardware interface for the application. For our system we selected a C2004A LCD screen. 
		It is identical to the Hitachi HD44780 screen used in the experiments. However, it has a bright, blue screen and allows more lines to be programmed. A library was created to interface with it quickly.
		
		\begin{figure}[H]
			\centering
				\includegraphics[scale=0.35]{img/LCD}
			\caption{C2004A LCD Screen Interfaced with the Tiva C, DHT11, and Strain Gauges.}
		\end{figure}
		
		
		
		\subsection{Wind Vane}
		
		The Wind vane is used to show the user the direction of the current air flow inside the tunnel. This component was not in our original design, but added later to the project by Dr. Jim\'{e}nez. The main purpose of the wind vane is to make AeroBal look more and more like a real wind tunnel with all of the major features found not only in other wind tunnels, but also in other places involving high levels of wind, such as farms.
		The wind vane is a rotating potentiometer moved by the vane tied to the rotating part begin hit by the wind. The wind coming from the fan should move the potentiometer's resistance value which can be measured with the ADC.
		One big disadvantage was that the smooth potentiometer we found did not behave in a linear way. This caused a need to characterize it in a specific way depending on the values the ADC read on the angles we wanted. 
		\begin{figure}[H]
			\centering
				\includegraphics[scale=0.35]{img/windVane}
			\caption{Wind Vane sensor used in AeroBal.}
		\end{figure}
		
		\begin{figure}[H]
			\centering
				\includegraphics[scale=0.35]{img/estimationDiagram}
			\caption{Paper with angles and values used to characterize the sensor.}
		\end{figure}
		
		In the software, the ADC values shown by the Tiva on each of the wanted angles were stored in an array in the sense of a lookup table. Whenever the wind vane is moved, the software searches through the lookup table to find the value in the array nearest to the one provided by the ADC and interpolates using the value in the position previous to the one found in the lookup table.
		
		The wind vane uses a 3.3V source. This is also provided by the Tiva, so interfacing with it was fairly simple as well in a hardware sense. The process of recording all the values inside an array, however, was long and tedious. 
		
		\subsection{Servo Motor}
		
		The servo motor in our system is used to control how much air is going inside the tunnel from the fan we provided. It is expected that the flow of air into the tunnel from the fan is controlled by the voltage applied to the fan. However, the wind tunnel in the Civil Engineering at UPRM uses a different method. \\
		Instead of controlling the voltage applied to the fan, they instead have windows inside the tunnel which they control instead. These windows control the air flow going into the tunnel fairly easily, without the hassle of working with the high voltages of the fan. \\
		In order to emulate the original tunnel as much as possible, we made our own windows. These windows can be controlled by hand, but to make it automatized we use a servo motor to control them instead. \\
		The servo motor we selected is the MG995 Standard Servo Motor. This motor works with a 5V source and can move at a speed of 0.20 seconds per 60 degrees. It's single function is to move the mechanical parts connected to the windows. 
		
		\begin{figure}[H]
			\centering
				\includegraphics[scale=0.3]{img/servoMotor}
			\caption{MG995 connected to windows in wind tunnel.}
		\end{figure}
		
		\subsection{Android Application}
		
		One of the features of AeroBal that most other wind tunnels do not have is a personal Android application used for controlling various aspects of the tunnel and obtaning the data from the experiment. With the Android application, the user may turn the fan on or off as desired, and view the measurements also displayed on the LCD screen in their own device. If extended, the tablet can do more analysis of the system and provide more visual data.
		
		\begin{figure}[H]
			\centering
				\includegraphics[scale=0.2]{img/androidApp}
			\caption{User interface for Android Application.}
		\end{figure}
		
		Although not a necessity, an Android application is a good way to show how up to date we want to make the wind tunnel, using the newest technologies available. 
		
		\subsection{Miniature Wind Tunnel}
		
		A miniature wind tunnel had to be made in order to test our interfaced devices, given that we don't have access to the real-sized wind tunnel in the department of Civil Engineering at UPRM. This miniature tunnel was made entirely of wood, and we integrated a fan into it as the wind source. 
		
		\begin{figure}[H]
			\centering
				\includegraphics[scale=0.3]{img/GroupPhoto}
			\caption{First finished version of the wind tunnel, alongside all 4 group members.}
		\end{figure}
		
		The tunnel itself took around 6 weekends to complete plus planning, polishing/conditioning sessions, and not counting the electronic parts added to it later. After the first version was completed, windows were added to the tunnel. These windows are meant to be operated with a servo motor. 
		
		\begin{figure}[H]
			\centering
				\includegraphics[scale=0.45]{img/Windows}
			\caption{Windows inside tunnel; controlled by servo motor.}
		\end{figure}
		
		After we included all the external physical components (windows and strain gauge bases), we moved the wind tunnel to the Microprocessor Interfacing Laboratory (MIL). Since then, all our testing has been conducted inside the lab and other components have been slowly integrated into it.
		
		\begin{figure}[H]
			\centering
				\includegraphics[scale=0.35]{img/WindTunnelMIL}
			\caption{Wind tunnel inside MIL; integration around 65\% done by then.}
		\end{figure}
		
		This tunnel is expected to be painted and refined to look presentable to a big crowd by the time the project is 100\% done. Although it is not the most important part of the project, it is the showcase for great presentations of it.
		
		