\section{Conclusion}
	
	In conclusion, we have successfully developed an embedded system that can make the research conducted in the Civil Engineering Department more productive by providing an system that automates the procedure being performed. The main objectives which were to measure the aerodynamic forces were achieved by the use of the strain gauge sensors which provide a very high presicion measurement of the forces to the order of 0.01 oz which is well below the required 0.01lb specified by Professor Zapata. The measurement of the forces is therefore more accurate and the balance designed even achieves adding a third component for measurement named side which measures the horizontal force perpendicular to the drag force thus providing complete measurement in free space. The use of strain gauges and its interface with the microprocessor eliminates most of the manual intervention by the experimenters. The design of the balance makes it portable and adaptable to the wind tunnel already present in the UPRM facility. Additionally, sensors to provide environmental data of the experiment, were interfaced such that all of this data which had to be manually measured could be provided automatically.

	The user interface provided through the Android application adds function to the system by allowing more real state to show and organize experiment data. It provides a base for extensions that can automate procedures normally performed by experimenters such as scaling the results to their life like components. 



