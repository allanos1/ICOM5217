\vspace*{\fill}
	\begin{abstract}
        Wind tunnels are used for researching aerodynamic characteristics which are the effects of moving air on solid objects. One of the characteristics studied is the force that air causes on object. To do this, the wind tunnel of the University of Puerto Rico, Mayag\"{u}ez Campus uses technologies that are dated and have to manually be performed, which introduces complexity and error into experiments conducted. In this report, a system that would improve the one previously described is presented. 
        The system would improve the current wind tunnel in an important way by adding sensors connected to a system that performs the procedure of obtaining the measurements automatically. The main sensor used in this system are strain gauges which allow the measurement of force electronically.These are integrated with a device that allows the forces to be transferred to the strain gauges. Additionally, different parameters of the experiment must be studied such as temperature, pressure, humidity, wind direction and speed, for which a sensor for each one is included in the system. To allow easier interaction, the system provides both a hardware LCD interface and a Bluetooth connected software mobile application that additionally performs storage of the data for later processing. To connect all of these components together, a Tiva TM4C123GH6PM Microcontroller is used as the controlling component of the system. AeroBal would make recording data much simpler than it currently is, and could help make research easier in the future for constant users of the tunnel, such as mechanical and civil engineering students.
	\end{abstract}
\vspace*{\fill}