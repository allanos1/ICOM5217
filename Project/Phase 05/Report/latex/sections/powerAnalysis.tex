\section{Power Analysis and Driver Compatibility}

	\subsection{Power Supply Requirements}
		\textit{AeroBal} will be a system that always stays in the same place. Therefore, we will use the voltage coming out of a regular power outlet as our main voltage source. In Puerto Rico, the voltage coming out of a standard wall outlet is 120V with a frequency of 60 Hz\cite{ref:pAnalysis1}. This would act as our \textit{unregulated voltage source}. \\

		120 Volts is excessive for the amount of peripherals we are going to use. Besides that, the recommended V$_{DD Rec}$ values of all of our components were are always one of the following three values: 3.3V, 5V, or 12V. \\

		Due to this problem, we though that it would be convenient to get a power supply that regulates the 120V AC from the outlet into the convenient values mentioned. After some research, we found a power supply that met our demands. This was the \textbf{Cooler Master eXtreme Power Plus 550-Watt ATX}, which comes at a cheap price and perfectly meets our voltage demands. \\ 

		This power supply gives us seven different output voltages through Molex Power connectors. These are summarized in the following table. \\



		\subsubsection{Voltage}

			\textit{AeroBal} uses a variety of peripherals that have different operating conditions. All of these will be presented in the following table. \\

			% Powers will be: 12, 5, 3.3

			% Anemometer: homemade, abanico de la computadora, 12 voltios, 3.5V a 24V (confirmar con datasheet)

			% Wind vane sensor: potentciometro, 3.3V, foto en el cel

			% Instrumentation amplifier: conectado al strain gauge ( el strain % gauge realmente se llama un load cell sensor).

			\begin{tabular}{|c|c|c|c|c|c|}
				\hline
					& V$_{DD Rec}$ & V$_{IH}$ & V$_{IL}$ & V$_{OH}$ & V$_{OL}$ \\
				\hline
					& & & & &  \\
					MCU (TM4C123GH6PM) & 3.3V & 2.48V & 1.16V & 2.4V & 0.4V \\
					Wind Speed Window Actuator (SM-42BYG011) & 12V & - & - & - & - \\
					Barometric Pressure Sensor (BMP085) & 3.3V & 2.15V & 0.66V & 1.32V & 0.3V \\
					%Anemometer Sensor (Self made) & 5V & - & - & - & \\
					Humidity and Temperature Sensor (DHT11)& 5V & - & - & - & - \\
					%Wind Vane Sensor (Self made) & 3.3V & - & - & - & \\
					Instrumentation Amplifiers (x6) (INA125) & 12V & 2V & 0.8 V & 1.4V & 0.8 V\\
					Bluetooth (JY-MCU) & 3.3V & 2V & 0.66V  & 2.64V & 0.66V \\
					LCD (C2004A) & 3.3V & 2.25V & 0.66V & 2.5V & 0.66V \\
				\hline
			\end{tabular} \\ \\

			The V$_{DD Rec}$ values of all of our components are either 3.3V, 5V, or 12V. The power supply that we chose gives us outputs for all three of these voltages. One of the main reasons we chose the chosen power supply was because of how convenient it would be to have a cheap power supply that gave us the exact voltages we needed for all of our components. \\

			It is also worth mentioning that there will be two extra components not in the list. They are not in the list because we will make them ourselves, so there are no data sheets for them. \\ 

			These components are: 
			\begin{itemize}
			  \item \textbf{Anemometer Sensor}
			  \item \textbf{Wind Vane Sensor}
			\end{itemize}  

			These components will be designed to work with an V$_{DD Rec}$ value of 3.3V for the wind vanea and a V$_{DD Rec}$ value of 5V for the anemometer. Thus, the final number of components in each one of the three used output power connectors will be: \\

			The number of components per connector are: 
			\begin{itemize}
			  \item \textbf{+12V connector} - 2 Components
			  \item \textbf{+5V connector} - 2 Component
			  \item \textbf{+3.3V connector} - 5 Components.
			\end{itemize} 

		\subsubsection{Current}

			The following table summarizes the values of the currents of each one of our components. 

			\begin{tabular}{|c|c|c|c|}
				\hline
				& I$_{In}$ &  I$_{Out}$ & I$_{Leak}$\\
				\hline
				& & &   \\
				MCU (TM4C123GH6PM) & 4mA & 4mA & 1$\mu$A  \\
				Wind Speed Window Actuator (SM-42BYG011) & 330 mA & - & -\\
				Barometric Pressure Sensor (BMP085) & 7.7 mA  & 1.5 mA & 5 $\mu$A \\
				%Anemometer Sensor (Self made) & 50mA & 50mA & 0.3$\mu$ \\
				Humidity and Temperature Sensor (DHT11) & 2.5 mA & 1mA & 150 $\mu$A \\
				%Wind Vane Sensor (Self made) - 33mA & 1 & 1 & 1\\
				Instrumentation Amplifiers (x6) (INA125) & 6 mA &  700$\mu$A & 2 nA  \\
				Bluetooth (JY-MCU) & 40 mA & - & - \\
				LCD (C2004A) & 150$\mu$A & 50$\mu$A & 1$\mu$A \\
				\hline
			\end{tabular} \\ 

			There are plenty of I$_{In}$ values for our components. Luckily, our power supply has a wide and comfortable amount of current in each output. This was another of the top reasons why we chose this supply source. \\

			The analysis of whether the current provided is enough for our components is as follows:

			\begin{itemize}
			  \item \textbf{+12V connector}(Capacity: 16A):
			  
			  This connector will have seven components. The sum of I$_{In}$ if these components is: $$ \sum_{i=1}^{7} c_i = 4mA + (6)(6mA) = 60mA. $$ $$60 mA < 16A $$ 
			  \item \textbf{+5V connector} (Capacity: 25A):
			  
			  This connector will only have one component, and it's value is greatly below our capacity: $$2.5 mA << 25A $$ 
			  
			  \item \textbf{+3.3V connector} (Capacity: 25 A):
			  
			  This connector will have six components. Two of them will be made by ourselves, so we don't know yet how much current they will need. The ones we do know are the following: $$\sum_{i=1}^{4} c_i = 4mA + 7.7 mA + 40 mA + 150 uA = 51.9 mA $$  $$51.9mA << 25A$$
			  
			  Although we are not considering the two remaining components, we are sure that the remaining current capacity of the connector will be more than enough to supply them without any problem.
			  
			  \textbf{NOTE:} Although 6 instrumental amplifiers were listed, we are studying a way of using only 3 instead. This would decrease the needed current out of the +12V connector.
			\end{itemize} 


		\subsubsection{Driving Capability}
			The value V$_{IH}$ of the GPIO pins of our MCU is 2.4V. This means that all the if components interfaced to are working as loads would need to have a V$_{IH}$ lower than 2.4V. This condition is met, given that the component closes to 2.4V is the LCD, which has a value V$_{IH}$ of 2.25V. \\

			The value of V$_{OL}$ of the GPIO pins of our MCU is 0.4V. This means that all the components interfaced to it working as loads would need to have an V$_{OL}$ above of 0.4V. This condition as also met, as the component with the lowest value of V$_{OL}$ in the other components is 0.66V.

		\subsubsection{Maximum Pin Current}

			The maximum current of the GPIO pins of our MCU depend on which juncture they are in, and their position in the board. To show the values of each, they will be presented in two tables. The first one describes where in the board they are located. The second shows the max current value per position in the board: \\

			\begin{tabular}{|c|c|}
			\hline
			 Side & GPIO's \\
			\hline
			&    \\
			Left & PB[6-7], PC[4-7], PD7, PE[0-3], PF4 \\
			Bottom & PA[0-7], PF[0-3] \\
			Right & PB[0-3], PD[4-5] \\
			Top & PB[4-5], PC[0-3], PD[0-3,6], PE[4-5] \\

			\hline
			\end{tabular} \\ 

			\begin{tabular}{|c|c|}
			\hline
			 Side & I$_{MAX}$ \\
			\hline
			&    \\
			Left & 30 mA \\
			Bottom & 35 mA \\
			Right & 40 mA \\
			Top & 40 mA \\

			\hline
			\end{tabular} \\ 	 

	\subsection{Voltage Compatibility}

		None of our components require alternating current (AC) Voltage to work. Given that we will use power supply that converts the AC voltage off the outlet into DC voltage, the output DC voltage should suffice for all peripherals. Thus, we should not have any compatibility issues regarding voltages between source and peripherals. 

	\subsection{Thermal Analysis}

		On thermal analysis we need to determine the maximum power dissipation the chip can handle. Using function 8.11 from the book:

				$$P_{diss(max)}=(T_{J(max)}-T_{A})/\theta_{JA}$$
		According to the datasheets:
				$$T_{J(max)} = 150 C$$
				$$T_{A} = 25 C$$
				$$\theta_{JA}=54.8 C/W$$
				$$P_{diss(max)}=(150C-25C) / 54.8 = 2.28 W $$
		Hence, 2.28 W is the maximum power dissipation of the microprocessor.

		The current required by the processor while running at full speed 80MHz with V$_{DD}$ = 3.3V and ambient temperature 25C is 45.1mA. The current required from GPIO devices is 51.9mA in total. The total current is then

			$$I_{DD(avg)} + \sum_{allpins}^{} c_i = 45.1mA + 51.9mA = 97 mA$$
		Multiply by voltage level V$_{DD}$ = 3.3V:
			$$P_{diss}= 3.3V * 97mA = 320 mW < P_{diss(max)}= 2.28 W $$
		Therefore our system complies with the thermal limitations of the chip and it is safe to use for the application.


	\subsection{Power Supply Capacity}

		For our project, we chose Cooler Master eXtreme Power Plus 550-Watt ATX as our power supply. The main characteristics of our power supply were shown at the start of the Power Analysis. As a reminder, they are as follows: 

		\begin{tabular}{|c|c|c|}
		\hline
		Source & Voltage & Current  \\
		\hline
		& & \\
		1 & +3.3V & 25A  \\
		2 & +5V & 25A  \\
		3 & +12V 1 & 16A  \\
		4 & +12V 1 & 16A  \\
		5 & -12V & 0.8A  \\
		6 & -5V & 0.8A  \\
		7 & +5VSB & 2.5A  \\
		\hline
		\end{tabular} \\

		There are various reasons why we chose this power supply. Some of these are: 

		\begin{itemize}
		  \item \textbf{Convenience} - All of our interfaced components use voltage sources of either 3.3V, 5V or 12V. This power supply gives us all three. 
		  \item \textbf{Price} - The price of our power supply without any discount is \$59.99. Online, the supply can be obtained at \$19.99 using discounts. We consider this to be a completely acceptable value for the power supply of our system. 
		  \item \textbf{Efficiency} - According to our power supply's website, it is 70\% better efficiency-wise at typical load operation.
		  \item \textbf{Risk Free} - We initially considered doing our own power supply for \textit{Aerobal}. However, once we discovered this power supply we thought that maybe it was better to buy one that was already made and worked instead of taking the risk of wasting more time trying to build a power supply. This is something we have never done before, and we prefer focusing in the interfacing aspects of the projects in order to make sure it is ready in time.

		\end{itemize}


\newpage
		 
% DO NOT ERASE
\leavevmode
