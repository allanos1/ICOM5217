\section{Future Work}

	The next step in the development on our system is to implement it where the current weighting scale in the Civil Engineering Department at UPRM. For the implementation of the system, the current noise cancellation method is software averaging and hardware averaging from the ADC, however hardware methods of cancelation may be explored to be able to obtain more reliable values. The method for obtaining the wind direction uses a potentiometer that causes friction to movement when other wind vanes use a magnetic implementation such that the friction is very close to none which could be adapted to our implementation. The wind speed calculation is done with an anemometer that does not have a reference or a confirmation value and for such reason another system to obtain wind speed could be implemented. Using two barometric pressure sensors could provide a differential measurement of pressure that allows the computation of the speed of the wind inside the tunnel.

	In software, the data provided is just the raw data obtained from the measurements of the procedure. There are additional parameters and procedures used for the data experimented such as averaging, determining outliers, graphing the data, comparison of experiments, visualization of data, experiment scaling, and others that can be implemented. The application developed in Android allows the use of a component that has the ability to expand its computing power for analysis. Hence the system software can be expanded to provide more features to the experimenters.