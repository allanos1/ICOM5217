\section{Introduction}
		Wind tunnels \cite{ref:intro1} are structures where an object is placed to study its aerodynamic
		characteristics. The tunnel pulls air into it to achieve laminar flow, and the motion of this air
		acts on the object, causing forces (drag, lift, side) and moments (pitch, roll, yaw) to act on it.
		To study large objects, scaled down versions of these models are placed inside the tunnel and the
		values are then scaled to their counterparts. 
		 
		Numerous institutions around the world use wind tunnels. 
		Some examples are NASA, the University of Maryland \cite{ref:intro2}, and the University 
		of Southampton \cite{ref:intro3}. The Civil Engineering Department of the University of 
		Puerto Rico, Mayaguez Campus \cite{ref:intro4} has a small 2-room wind tunnel used for research. 
		In the past, this tunnel has been a central part of analyzing the aerodynamics of numerous
		projects, such as a Solar Car, ailerons, and tanks of storage for fluids.It was constructed in 
		1983 and since then it has not been improved much.
		
		%%%%
		%The conditions of the laboratory 
		%containing the wind tunnel can be considered of poor quality by today’s standards. 
		%For example, two of the computers in the Wind Tunnel Laboratory have 16 MB of on-board 
		%memory, and 424 MB of hard drive space. Another component of the tunnel which is in 
		%deplorable conditions is also one it’s main parts. This component is the 
		%mechanical balance that holds the model being analyzed. The balance of this particular
		%tunnel is composed various steel bars connected together. These bars have halves of 
		%plastic milk bottles hanging from the bars.  Figure 1 shows an image of the mechanical balance.
		%%%%


		The technology used in the tunnel is purely mechanical and requires manual intervention by 
		the experimenter. To measure force the tunnel uses a balance that allows free movement in 
		the horizontal and vertical axis, using a design that connects various steel bars together.
		Below we can see an image of the design implemented in the Civil Department Wind Tunnel: 

		\begin{figure}[H]
			\centering
				\includegraphics[scale=0.7]{img/intro-1}
			\caption{Current Mechanical Balance used in the wind tunnel.}
		\end{figure}

		Whenever an object is placed in the tunnel and the wind tunnel is started, the object tilts the
		balance out of equilibrium. An example of the tilted balance can be seen in Figure 2. The direction 
		of the tilt is caused by the drag and lift forces acting on the model. The data from the balance is 
		obtained by pouring sand on whichever cup needs it in order to balance the device until 
		it in equilibrium again. Then the user takes the plastic cup and weights the poured sand to obtain
		the force magnitudes.

		\begin{figure}[H]
			\centering
				\includegraphics[scale=0.7]{img/intro-2}
			\caption{Mechanical Balance Tilted.}
		\end{figure}
		
		The process of obtaining the data using the mechanical balance of this particular wind tunnel 
		can be time consuming and error prone due to this manual intervention. We propose an electronic system 
		to renew the mechanical balance, having a microprocessor as a central point, in order to 
		improve wind tunnel as a whole. Our model would modernize the current tunnel in a 
		meaningful way without having the economic implications of buying an entire new system.
