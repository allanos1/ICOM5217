\section{Theoretical Background}

	\subsection{Brief Historical Background}
	
	Wind tunnels date back to 1871. These first tunnels were mostly small and enclosed, used to test aerodynamic characteristics of objects. One of the most popular wind tunnels from the 1900's was made by the Wright Brothers in their development of their Wright Flyer \cite{ref:wright1} as can be seen on Figure 3.
	%http://wright.nasa.gov/airplane/tunnel.html
		
		% from http://www.nationalmuseum.af.mil/shared/media/photodb/photos/051019-F-1234P-003.jpg
		\begin{figure}[H]
			\centering
				\includegraphics[scale=0.9]{img/wrightBrothersTunnel}
			\caption{Wright Brother's Wind Tunnel \cite{ref:wright2}}
		\end{figure}
		
		% from http://www.centenary.edu/attachments/physics/tmessina/phys301-2007/windtunnel2.png
		\begin{figure}[H]
			\centering
				\includegraphics[scale=0.5]{img/windtunnel2}
			\caption{Example of a simple wind tunnel.}
		\end{figure}
	
	Progress and importance of wind tunnels increased during the Second World War. One of the biggest tunnels ever constructed was made in Wright Field in Dayton, Ohio in 1941. \cite{ref:wright3} 
	%https://www.asme.org/getmedia/5fe3daaf-75a3-4eb8-b5a7-da95fdc2413e/183-Wright-Field-5-Foot-Wind-Tunnel.aspx
	This tunnel had measurements of 45 ft in height and around 20 feet in width. Large aircraft models were tested here at speeds of around 400 mph. By the end of World War Two, the United States had built various tunnels, including an even bigger one in Moffett Field near Sunnyvale, California. Since then, a great number of institutions around the world have built their own wind tunnels, including the University of Puerto Rico.
	
	\subsection{How It Works}
	
	A wind tunnel consists of three main parts: (1) the wind source, (2) the studied object, (3) the sensors to obtain the measurements.

	\subsubsection{Wind Source}

	The wind source of a wind tunnel can come from different devices. Small tunnels like AeroBal use a simple conventional fan found in any discount store. It's speed does not exceed 25 mph. Larger wind tunnels used in major organizations use fans with speeds higher than 100 mph. Special types of wind tunnels, called Supersonic Wind Tunnels, can produce supersonic speeds (around 345 mph).  	
	
	\subsubsection{Studied Object}
	
	Arguably the most important element of a wind tunnel is the object being studied in it. In small wind tunnels, the objects are usually replicas of larger versions of that object. Objects such as cars, helmets, rockets, and airplanes, and which are normally under the effect and dependent of the motion of air, are studied.
	
	% from http://www.nasa.gov/centers/ames/images/content/707247main_11x11%20test.jpg
		\begin{figure}[H]
			\centering
				\includegraphics[scale=0.225]{img/nasaAirplaneWT}
			\caption{Miniature Airplane used inside a NASA Wind Tunnel. \cite{ref:wright5}}
		\end{figure}
	
	\subsubsection{Sensors}
	
	The main sensor used in the tunnel is the weight sensor that measures the forces. To measure the force of drag and lift, only the force while under the effect of the motion of air must be calculated. There are two stages for an experiment: 1) calibrate the instrument to equilibrium with no wind, 2) with the wind turned on and causing the instrument to not be in equilibrium, return the instrument to equilibrium using some form of weight which can be measured. Therefore, the weight used to return to equilibrium represents the measurements of drag and lift, depending on where they are placed in the instrument, to counter the components of them:

	$Component_{object} = Weight_{Wind} - Weight_{NoWind} $


	Should the object use a \emph{base} for the object which is considerably affected by the wind, then the base's components (drag and lift) must be measured using the same procedure, such that these are substracted from the end results which contain a combined measurement of both the base and the object. 

	$Component_{object} = Component_{Combined} - Component_{Base}$


	The weight sensor at the UPRM Wind Tunnel is implemented by manually pouring sand, and then using a weight scale to measure its weight. Other factors measured with sensors are temperature, pressure, relative humidity, wind direction, and wind speed. These can be implemented with different sensors of each type and are measure before and during the wind tunnel's operation.

	% from http://itll.colorado.edu/images/uploads/test_measurement_equipment/sensors.jpg
	%\begin{figure}[H]
	%	\centering
	%		\includegraphics[scale=1.2]{img/forceSensorWT}
	%	\caption{Example of force sensors used in a wind tunnel.}
	%\end{figure}

		

		