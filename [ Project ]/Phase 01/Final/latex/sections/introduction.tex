\section{Introduction}
		Wind tunnels \cite{ref:intro1} are large tubes where an object is placed, and air is blown from 
		powerful fans in a way that the air passes said object. The behavior of the object 
		varies, depending on the aerodynamic characteristics of it. Therefore, we may say 
		that wind tunnels are good tools to simulate the behavior of the objects in free 
		flight. Some examples of objects that might be placed in wind tunnels are scaled-down 
		versions of new airplane models, or cars with and without spoilers (in order to test
		how effective the spoilers are). Numerous institutions around the world use wind tunnels. 
		Some examples are NASA, the University of Maryland \cite{ref:intro2}, and the University 
		of Southampton \cite{ref:intro3}. The Civil Engineering Department of the University of 
		Puerto Rico, Mayaguez Campus \cite{ref:intro4} has a small wind tunnel used for research. 
		In the past, this tunnel has been a central part of analyzing the aerodynamics of numerous
		projects, such as a Solar Car, ailerons, 
		and tanks of storage for fluids. Unfortunately, this particular wind tunnel was constructed in 1983, and renewal of it has not been done. The conditions of the laboratory 
		containing the wind tunnel can be considered of severely poor quality by today’s standards. 
		For example, two of the computers in the Wind Tunnel laboratory have 16 MB of onboard 
		memory, and 424 MB of hard drive space. Another component of the tunnel which is in 
		deplorable conditions is also one it’s main parts. This component is the 
		mechanical balance that holds the model being analyzed. The balance of this particular
		tunnel is composed various steel bars connected together. These bars have halves of 
		plastic milk bottles hanging from the bars.  Figure 1 shows an image of the mechanical balance.

		\begin{figure}[H]
			\centering
				\includegraphics[scale=0.7]{img/intro-1}
			\caption{Current Mechanical Balance used in the wind tunnel.}
		\end{figure}

		Whenever an object is placed in the tunnel and the wind is blown, the balance is tilted. An
		example of the tilted balance can be seen in Figure 2. The direction of the tilt depends 
		on the characteristics of the model being used. The data from the balance is obtained 
		by pouring sand on whichever cup needs it in order to balance the device until 
		it is close to 90 degrees, as it was from the start. After the desired angle has been 
		achieved, the user takes the plastic cup and weights the poured sand. Calculations are 
		then made using the measurement and then the process of getting the needed data can continue.

		
		\begin{figure}[H]
			\centering
				\includegraphics[scale=0.7]{img/intro-2}
			\caption{Mechanical Balance Tilted.}
		\end{figure}

		It is not hard to imagine how tedious and inconvenient the process of obtaining the data using
		the mechanical balance of this particular wind tunnel is. Science fair projects from recent
		years have tried to make solutions to improve the tediousness of the mentioned process, all of 
		them to no avail. We propose an electronic system to renew the mechanical balance, having a 
		microprocessor as a central point, in order to significantly improve wind tunnel as a whole.
		Our model would modernize the current tunnel in a meaningful way without having the economic
		implications of buying an entire new system.