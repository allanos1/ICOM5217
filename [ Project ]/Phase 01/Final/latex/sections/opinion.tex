\section{Expert Opinion}
	ELECTRONIC BALANCE FOR EVALUATION OF AERODYNAMIC FORCES ACTING ON MODELS TESTED AT THE WIND TUNNEL LABORATORY

	The main application user for this project will be the students, faculty and general public that come to perform research work at the Wind Tunnel Laboratory of the Department of Civil Engineering and Surveying. The Supervisor of this facility and direct contact for technical expertise of the current project is Associate Director of the Department of Civil Engineering, Dr. Raul E. Zapata-Lopez. 
	
	Dr. Zapata presented us with the idea of working on the automation of one of the current measurement 
	devices of the laboratory. A mechanical balance system is used to measure aerodynamic (drag and lift) forces acting on objects or scaled models that are aerodynamically tested in a wind tunnel facility that the department developed many years ago. Using the momentum concepts (moment = force * distance), the aerodynamic forces acting on a given model are measured by seeking to balance the aerodynamic force by another force acting in opposite direction at a given distance from a pivot point. That force is obtained by pouring dry and loose sand into a bucket until the balance system is leveled to the original position before the aerodynamic forced started to act over the model. Because the drag force acts horizontally and the lift force acts on the vertical axis, the balance system works in those two directions. The equilibrium position is reached when all arms of the balance systems are leveled which are verified by the use of bubble levels. The sand is then collected and weighted on an external balance. The process is repeated three to five times or until the statistics of collected data is reasonable to explain the aerodynamic model behavior. The operation of this mechanical system is working effort intensive and requires at least two persons to perform as expected. 

	The proposed project seeks to transform that mechanical scale which requires manual interaction, sand and moving parts to be substituted by an automated system that will make the repetitive process of the tests easier to handle. 

	The current scale was built by Dr. Zapata who as a result knows all the factors in building the scale that could affect the experiment. To be able to replicate the behavior in an automated manner we will be contacting him to provide us with technical specifications of what the system must achieve, and at the end will decide if the chosen design complies with the specifications. As stated by the laboratory supervisor:

	"If possible, I would like to see the construction of an electronic scale that replaces the current mechanical scale in two directions. I am available to discuss any technical specification about the scale as well as any theoretical information. This project would enable us to run tests faster and more effectively. One of the main factors that the current project can help us is time effort reduction since the amount of time used for running the experiments is extensive. Another factor is the expected consistency of the collected data will keep and most probably improve the precision of the collected data.”
 
 	For any issues or questions Dr. Zapata can be contacted by email at raul.zapata@upr.edu and at the telephone number 787-265-3815 or 787-832-4040 (ext. 3815, 3434, 3559 at Dept. office and ext. 3376 at Wind Tunnel Lab.)

	Raul E. Zapata Lopez, PhD, PE, RPA
	Civil Engineering Wind Tunnel Lab Supervisor and
	Associate Director of the Department of Civil Engineering \\
	
	Dr. Eduardo Ortiz-Rivera (Professor of  the University of Puerto Rico, Mayagüez Campus), had the following comments:
	
	\begin{enumerate}
		\item -	An important characteristic of the system should be to use two accelerometers and compare values of both, in order to get the best precision possible. A gyroscope should be considered as well.
		\item - Bluetooth or some other of wireless communication to an external device has to be a part of the project. If the idea is to make the researcher have a more convenient system, then having to bend down to collect the data through something like an SD Card should be evaded.
		\item -	Value of the device should be considered; if it is low enough, then it could be marketable.
		\item - The system currently used should not be thrown away. Although it is not convenient to use, it is reliable in the sense that it can always be used in case our system malfunctions, plus it does not need electricity to work. 

 
 		
	\end{enumerate}
	

