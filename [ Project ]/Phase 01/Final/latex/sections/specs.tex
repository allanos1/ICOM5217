\section{Specifications}
	\subsection{Requirements, Features and Limitations}
		\begin{enumerate}
			\item The system must provide amount of force being applied to the object both in vertical and horizontal orientation when: 
			
			\begin{itemize}
				\item The object is not under aerodynamic effects to calibrate and establish reference (initial) values of force measurement.
				\item The object is under aerodynamic effects to determine the actual force.
			\end{itemize}
			
			\item The system must provide an interface to start experiments and take measurements (LCD prompts and buttons for example).

			\item The system must have a precision of at least the hundredths of pounds (0.01 lbs).

			\item The system must be able to change between units (newtows, pounds) on the results provided.

			\item The system must show the measurement of force in a simple GUI, the measurement of the coefficient of drag and coefficient of force, and show the average and variance of data.

			\item A limitation is that due to turbulence, the instant forces being measured might not provide consistent results. Hence the measurements must be averaged and the variance of data must be provided.

			\item The system must connect to multiple sensors that read the environmental data of the wind tunnel, such as humidity, temperature, and pressure, and provide this data.

			\item The system must be able to read and control the wind speed of the tunnel.

		\end{enumerate}


	\subsection{Hardware and Software Requirements}
	\textbf{Hardware}:
		\begin{enumerate}
		
 		\item Strain Gauges - The strain gauges will be used to measure the forces being applied to the object. This data will be transferred to the microcontroller.

 		\item Communications Module - The communications module will be used to send data to a remote user interface for data logging purposes.

		\item LCD - The LCD will be used to display data and user feedback as well as a menu to the user for setup.

		\item LED - The LEDs will be used to display system status.

		\item Buttons - The buttons will be used for user input in the menus provided through the LED and for control.

		\item A minimized version of the wind tunnel for testing. This tunnel will be useful as a testbench for the project to not interfere in the actual system.

		\item A mechanical system tied to the strain gauges that measures the amount of force applied. This mechanical system must move freely in the horizontal and vertical directions as minimum requirement. It must provide a minimally-obstructive extension that is inserted into the wind tunnel in order to provide a base for the object while interfering as little as possible with the aerodynamic measurements.

		\item A circuit that controls the actuator that controls the windows that change the wind speed of the tunnel. 

	\end{enumerate}
	\textbf{Software}:
	Various functions must be developed. They must, at a minimum, perform the following tasks: 
		\begin{enumerate}
			\item Store calibration values when the wind tunnel is off to use as reference (static values) and then substract them from actual or dynamic values when the wind tunnel is on to provide the correct measurement of force.
			\item Store strain gauge sensor data on user input.
			\item Compute statistical values of the results.
			\item Calculate aerodynamic parameters based on stored data.
			\item Communicate with user interface to display results.
			\item Check status of the system to follow operating conditions (wind speed).
			\item Display users the current procedure state of the system and available options.
			\item Prompt users to continue other procedure states with the appropiate message.
			\item Store and apply configuration parameters of experiment (wind speed, units, etc.). 
			\item Provide the control signal to change the wind speed of the tunnel when configuration is changed.
			\item Display status of the sensors that read environmental data.
		\end{enumerate}


	\subsection{Components}
		\subsubsection{Communications}
			The system will communicate with an application running on a tablet device that reads, interprets and displays the result. This communications interface is wireless using a Bluetooth module to achieve this communication.

		\subsubsection{User Interface}
			The system will incorporate a user interface in hardware and software. This UI will allow the user to initiate data reading and storage. While reading and when stopped,
			the UI will show the user the data in tabular and graph form, and display statistical information. For hardware, the system will display through a wired interface to an LCD the results of the experiment. For software, the system will communicate with a remote UI (tablet interface), through wireless communication using Bluetooth, which will read the values to display the tabular and graph form of the results.

		\subsubsection{Control Scheme}
			The system must control all the components that contribute to keeping the system in
			equilibrium. This must happen whether it is under the aerodynamic test or not. The
			system will also provide manual control to the balance orientation or tilt.

		\subsubsection{Microprocessor-Based}
			The microprocessor is a necessary component due to the need of providing control as a function of the sensor readings to maintain equilibrium in the system. The microprocessor is also in charge of displaying the information read from the sensors (force) to the user interface (hardware and/or) and to provide control from user input (hardware and/or software). In software UI, the microprocessor is also in charge of managing the communication from the physical system.