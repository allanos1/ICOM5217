\section{System Conception}
	\subsection{Global System View}

		\begin{figure}[H]
			\centering
				\includegraphics[scale=0.40]{img/globalview}
			\caption{Global System View of the proposed solution (tentative)}
		\end{figure}

		The wind tunnel balance system will be a system that will be able to calibrate itself for initial use using the strain sensor components which do not require the system to move like the actual system. This will provide a reference value from which the system will then calculate the difference to obtain the actual result. These actual results are used for two stages of the system: the setup stage and the experiment phase. For the setup phase, the experimenter needs to determine using the wind tunnel the aerodynamic properties of the base that will be used to place the object for which we want to determine the same properties. In the experiment phase, the object is placed in the base, and at this time, the system knows the aerodynamic properties of the base. Once the properties are obtained with both the object and the base, all the system has to do is a difference of the values obtained in the setup phase and the experiment phase.

		Aerodynamic properties determined by the system are the drag and lift forces acting on the object. Additionally, formulas can be applied using these forces to determine the coefficients of drag and lift which help with computations done when scaling objects that are minimized. This can all be provided using the UI of the system hardware (LCD) as well the application UI (on software). 


	\subsection{User Interface Level}

		The system UI proposed will provide users the convenience of reading the current measurements and logging this information onn user input. The UI will also provide the user with the ability to control the system. Two proposed user interfaces are to be implemented: the software UI using a personal computer, and the hardware UI using LCDs and LEDs.

		For the software UI, the user will be presented with an interface that displays the configuration options, the control options, the data being captured if the experiment is running, and the data for previous experiments.Using the information read, measurements can be stored on a database, experiment and statistical computations can automatically be performed from, these measurements and graphs can be provided. A mockup is shown in figure 5.

		\begin{figure}[H]
			\centering
				\includegraphics[scale=0.40]{img/UI-PC}
			\caption{UI Mockup}
		\end{figure}

		Computations that can be performed are for the coefficient of drag and lift which are calculated using the forces obtained from the strain gauges. This user interface will be updated periodically to display the status of the system, and the configuration settings allow the user to setup the desired wind speed as well as the desired units. 

		Additional to providing the display of information, the UI will provide the user with software control of the system. The UI will have buttons that will be analogous to the hardware buttons given to the system. These buttons will provide control over the initial setup of the system (Initial Setup), and the measurement of the drag and lift forces of the experiment (Measure). 

		For the hardware, UI and LCD panel will be provided which will display the current drag and lift forces acting on the object placed in the system. The UI will provide control over the initial setup phase, the experiment phase, and the configuration. The measurements will be stored as it is done in the software UI to provide an average and a variance. To achieve this, the hardware UI will then feature a menu controllable by buttons on the LCD. 